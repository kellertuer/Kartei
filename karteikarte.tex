%
%		* ----------------------------------------------------------------------------
%		* "THE BEER-WARE LICENSE" (Revision 42/023):
%		* Ronny Bergmann <mail@darkmoonwolf.de> wrote this file. As long as you retain 
%		* this notice you can do whatever you want with this stuff. If we meet some day,
%		* and you think  this stuff is worth it, you can buy me a beer or a coffee in return. 
%		* ----------------------------------------------------------------------------
%
%
% Dokument-Vorlage für Din A6 Karteikarten 
% -- Version 1.5 --
%
% - Karteikartenumgebung mit automatischer Durchnummerierung (Referenzen verweisen auf die Karteikarten # Nummer)
% - Für eine Karte kann
% 	* Kopf oben rechts und rechts (wird jeweils kursiv gesetzt)
% 	* Ein Titel für die Vorderseite angegeben werden
% 
% - Karteikartenumgebung karte die wie folgt verwendet wird
% 
%	\begin{karte}{KopfRechts}{KopfLinks}{Karteikartentitel}
%		Rückseite
%	\end{karte}
%
\documentclass[twoside, 10pt, landscape]{article}
\usepackage[utf8]{inputenc} %UTF8
\usepackage[OT1]{fontenc}
\usepackage[scaled]{helvet}
\usepackage[ngerman]{babel} % Neue Rechtschreibung
%Menge an Paketen die für vor allem mathematisches und informatisches gut brauchbar ist
\usepackage{graphicx,textcomp,booktabs,mathptmx,courier,calc,fancyhdr,trfsigns}
\usepackage{enumerate,subfigure,listings,color,amsmath,amssymb,euscript}
%
% Definieren des A6-Blattes mit entsprechenden Schriftgrößen
%
\usepackage{geometry}
\geometry{	a6paper,%
			tmargin=1cm, bmargin=.5cm,% 
			lmargin=.5cm, rmargin=.5cm,% 
			headheight=1.3em, headsep=1em,% 
			footskip=.25cm} 

%
% Deutsche Absatzformatierung
%
\setlength{\parindent}{0pt} 
\setlength{\parskip}{1em}	

%
% Standard-Layout: 	Kopf:  Kartennummer vorne mittig, hinten links (also bei y0), hinten mitte „Antwort“
% 					Fußzeile: leer
%
\renewcommand{\headrulewidth}{0.2pt}
\renewcommand{\footrulewidth}{0.0pt} 
\pagestyle{fancy}{% 
	\fancyhf{}%
	\fancyhead[CO,LE]{\emph{\theCardID}}%	
	\fancyhead[RE]{}%
	\fancyhead[CE]{\emph{Antwort}}%
}

%
% Karteikartenzähler & Ausgabe des Counters in Form „# Nummer“ bei Referenzen und in der Kopfzeile
%
\newcounter{CardID}
\renewcommand{\theCardID}{\# \arabic{CardID}}

%
%	Karteikartenumgebung:
%	#1 Kopfzeile vorne Links
% 	#2 Kopfzeile vorne rechts
% 	#3 Titel der Vorderseite  	
%
\newenvironment{karte}[3]
{% Vor der Umgebung: Vorderseite bauen
 %
	\pagestyle{fancy}{% 
		\fancyhead[LO]{\emph{#1}} %Fach
		\fancyhead[RO]{\emph{#2}} %Kommentar
	}%
~\vfill{~\hfill \Large #3\hfill~}\vfill~
\refstepcounter{CardID}
\newpage%
}%
{% Nach Umgebung: Warnung im Kaop (bis TODO gelöst) 
 % und neue Karteikarte auf ungerader Seite beginnen lassen
 %
	\fancyhead[LO]{\emph{TODO - ab und an Karteikarte zuviel generiert,}}
	\fancyhead[RO]{\emph{wenn Rückseite recht voll}}
	\cleardoublepage
}

%
%	Beginn des Dokumentes, hier mit einer recht vollen Karte und 
% 	2 Beispielkarten mit Referenzbeispiel
%

\begin{document}
	
	\begin{karte}{Rechnerarchitektur}{Prüfungsfrage}{RISC \& CISC}
			\begin{tabular}{p{.45\textwidth}|p{.49\textwidth}}
				\parbox[t]{.45\textwidth}
				{\textbf{CISC} - Complex Instruction Set Computing\\
				\begin{itemize}
					\item klassischer Befehlssatz
					\item Ziel: komplexität in die CPU $\Rightarrow$ Speicherersparnis
					\item viele leistungsfähige Einzelbefehle
					\item unterschiedliche Befehlsformate/breite
					\item Nachteile: hoher Decodierungsaufwand, längere Einzelausführungszeit
					\item ursprünglich mikroprogrammiert
					\item \emph{Beispielarchitektur} x86\\
					(ab Pentium Pro interne Umsetzung in RISC/ $\mu$Op)
				\end{itemize}}
				&
				\parbox[t]{.45\textwidth}
				{\textbf{RISC} - Reduced Instruction Set Computing\\
				\begin{itemize}
					\item reduzierter Befehlssatz, einheitliche Befehlsbreite
					\item kurze Decodierzeit \& schnelle Einzelbefehle
					\item[$\Rightarrow$] schnellere Interrupts 
					\item nur \emph{Load} und \emph{Store} greifen auf Speicher zu
					\item[$\Rightarrow$] großer Registersatz 
					\item führt die SPEC-Top500 an
					\item teilweise trotzdem komplexere Befehle etwa MMX oder AltiVec
					\item \emph{Beispielarchitektur} PPC, ARM, ATMega
				\end{itemize}
				}
			\end{tabular}
	\end{karte}

	\begin{karte}{Gar kein Fach}{LabelKarte}{eine Karte}
		Eine Karte mit einem Label \label{label1}
	\end{karte}

	\begin{karte}{Auch kein Fach}{Referenzkarte}{noch eine Karte}
		Und ein Verweis auf die LabelKarte (siehe \ref{label1}).
	\end{karte}

\end{document}