%
%		* ----------------------------------------------------------------------------
%		* "THE BEER-WARE LICENSE" (Revision 42/023):
%		* Ronny Bergmann <mail@darkmoonwolf.de> wrote this file. As long as you retain 
%		* this notice you can do whatever you want with this stuff. If we meet some day,
%		* and you think  this stuff is worth it, you can buy me a beer or a coffee in return. 
%		* ----------------------------------------------------------------------------
%
%
% Beispiel zur Dokumentvorlage für Din A6 Karteikarten 
% -- Version 1.8 --
%
\documentclass[a6paper,14pt,print,grid=both]{kartei}

\usepackage[utf8]{inputenc} %UTF8
\usepackage[OT1]{fontenc}
\usepackage[scaled]{helvet}
\usepackage[ngerman]{babel} % Neue Rechtschreibung

\begin{document}
	\begin{karte}[Lebensphilosophie]{Wie lautet die Antwort auf die Frage nach dem Leben dem Universum und dem Ganzen Rest ?}[prüfungsrelevant]
	42
	ich bin allerdings eine Karte die viel zu voll ist 
	
	Dann hast Du meine Antwort nicht verstanden, denn man kann auch in diesem
	Fall Schriftattribute im optionalen Argument von item nutzen und auch dann
	bei KOMA-Script die Schrift für das Element descriptionlabel ändern. In dem
	Fall dann eben nicht, um fett dazu zu bekommen, sondern im fett los zu
	werden.
	
	Ronny Bergmann <mail@darkmoonwolf.de> wrote this file. As long as you retain 
	 this notice you can do whatever you want with this stuff. If we meet some day,
	 and you think  this stuff is worth it, you can buy me a beer or a coffee in return.
	
	Und noch ein Absatz.
	
	Und dann muss diese Karte doch endlich einmal voll sein...
	
	Die zweite Lösung (die mit report also ohne KOMA-Script) ist bereits eine,
	die temporär arbeitet. Wie man im ersten Fall ebenfalls temporär arbeitet,
	habe ich ebenfalls bereits erwähnt.
	\end{karte}
%	
	\begin{karte}[Zahlenkunde]{Was ist der Unterschied in der Verwendung von Drölf und $n$ bei „Ihnen“ ?}
	$n$ wird verwendet für Zahlen bis hin zu „verdammt groß“, Drölf nur bis hin zu verdammt.
	\end{karte}
%
	\section*{Informatik}
	\subsection*{Spaß mit Verweisen}
	%Den Kommentar im Stil ändern
	\renewcommand{\kommentarstil}{\textsc}
%
	\begin{karte}{Was ist verschränkte Rekursion ?}[ein Beispiel für Label]
	\label{karte:antwort} Siehe Karte \ref{karte:frage}
	\end{karte}
%
	\begin{karte}{Was ist die Antwort auf Karte \ref{karte:antwort} ?}
	\label{karte:frage}		Hier kommt man eigentlich gar nicht hin. Hier gibt es also nichts zu sehen, bitte blättern sie unauffällig weiter.
	\end{karte}
	\begin{karte}{nocheine}
		AB
	\end{karte}
\end{document}