%%
%    * ----------------------------------------------------------------
%    * "THE BEER-WARE LICENSE" (Revision 42/023):
%    * Ronny Bergmann <mail@rbergmann.info> wrote this file. As long as
%    * you retain this notice you can do whatever you want with this
%    * stuff. If we meet some day and you think this stuff is worth it,
%    * you can buy me a beer or a coffee in return.
%    * ----------------------------------------------------------------
%
%
% A german example using the Kartei.cls - including print and toc as
% options, hence all pages are Din A4.
%
% Last Change: Kartei 1.9, 2012/01/04
%
\documentclass[a6paper,10pt,grid=front%
,toc
%,print
]{Kartei/kartei}
\usepackage[utf8]{inputenc} %UTF8
\usepackage{hyperref}
\begin{document}
  \setcardpagelayout
  \begin{karte}[Lebensphilosophie]{Wie lautet die Antwort auf die Frage nach dem Leben dem Universum und dem Ganzen Rest ?}[prüfungsrelevant]
  42
  \end{karte}
  \begin{karte}[Zahlenkunde]{Was ist der Unterschied in der Verwendung von Drölf und $n$ bei Ihnen?}
  $n$ wird verwendet für Zahlen bis hin zu verdammt groß, Drölf nur bis hin zu verdammt.
  \end{karte}
%  \section{Informatik}
%  \subsection{Spaß mit Verweisen}
%  %Den Kommentar im Stil ändern
  \renewcommand{\kommentarstil}{\textsc}
  \begin{karte}{Was ist verschränkte Rekursion ?}[ein Beispiel für Label]
  \label{karte:antwort} Siehe Karte \ref{karte:frage}
  \end{karte}
%  \subsection{Spaß mit Verweisen II}
  \begin{karte}{Was ist die Antwort auf Karte \ref{karte:antwort} ?}
  \label{karte:frage}
    Hier kommt man eigentlich gar nicht hin. Hier gibt es also nichts zu sehen, bitte blättern sie unauffällig weiter.
  \end{karte}
\end{document}